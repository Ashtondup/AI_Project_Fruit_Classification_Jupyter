\documentclass[conference]{IEEEtran}
\IEEEoverridecommandlockouts
% The preceding line is only needed to identify funding in the first footnote. If that is unneeded, please comment it out.
\usepackage{cite}
\usepackage{amsmath,amssymb,amsfonts}
\usepackage{algorithmic}
\usepackage{graphicx}
\usepackage{textcomp}
\usepackage{xcolor}
\def\BibTeX{{\rm B\kern-.05em{\sc i\kern-.025em b}\kern-.08em
    T\kern-.1667em\lower.7ex\hbox{E}\kern-.125emX}}
\begin{document}

\title{Documentation of Fresh fruit detection AI}

\author{
\IEEEauthorblockN{1\textsuperscript{st} Bernard Swanepoel}
\IEEEauthorblockA{Faculty of Natural\\
and Agricultural Sciences\\
North-West University\\
39909476\\
Email: 39909476@mynwu.ac.za}\\   %<------ Line breaks in the current column
\and
\IEEEauthorblockN{2\textsuperscript{nd} Ashton du Plessis}
\IEEEauthorblockA{Faculty of Natural\\
and Agricultural Sciences\\
North-West University\\
34202676\\
Email: 34202676@mynwu.ac.za}\\[0.9cm]  %<------- Extra vertical space
\and
\IEEEauthorblockN{3\textsuperscript{rd} Nico Deng}
\IEEEauthorblockA{Faculty of Natural\\
and Agricultural Sciences\\
North-West University\\
33700710\\
Email: 33700710@mynwu.ac.za}\\                 %<-----------
}
\maketitle

\begin{abstract}
This paper presents a convolutional neural network (CNN) model and a mobile application for classifying fresh and rotten fruits. The CNN model is designed to distinguish between fresh and rotten apples, bananas, and oranges using image data. The model architecture consists of multiple convolutional layers, batch normalization, dropout, and fully connected layers. The training process utilizes data augmentation techniques, such as random horizontal flipping and normalization, to enhance the model's performance. The model is trained on a dataset of fruit images and evaluated for its classification accuracy. Additionally, a mobile application is developed to integrate the trained model for real-time fruit classification. The application allows users to load fruit images, preprocess them, and obtain predictions from the CNN model. The predicted class, either fresh or rotten fruit, is displayed to the user along with the corresponding image. The proposed system demonstrates the potential for automating fruit quality assessment and facilitating efficient sorting and grading processes in the agricultural industry or daily use.
\end{abstract}

\begin{IEEEkeywords}

\end{IEEEkeywords}

\section{Introduction}
\subsection{Brief overview of the project}
\subsection{Motivation and objectives}
\subsection{Importance of fruit classification and its applications}

\section{Related Work}
\subsection{Dataset and Preprocessing}
\subsection{Model Architecture}
\subsection{Training Process}

\section{Methodology}

\section{Experimentation and Results}

\section{Ease of use}
\subsection{Evaluation and Results}

\section{Conclusion}
\subsection{Maintaining the Integrity of the Specifications}



\section*{Acknowledgment}

The preferred spelling of the word ``acknowledgment'' in America is without
an ``e'' after the ``g''. Avoid the stilted expression ``one of us (R. B. 
G.) thanks $\ldots$''. Instead, try ``R. B. G. thanks$\ldots$''. Put sponsor 
acknowledgments in the unnumbered footnote on the first page.

\section*{References}

Please number citations consecutively within brackets \cite{b1}. The 
sentence punctuation follows the bracket \cite{b2}. Refer simply to the reference 
number, as in \cite{b3}---do not use ``Ref. \cite{b3}'' or ``reference \cite{b3}'' except at 
the beginning of a sentence: ``Reference \cite{b3} was the first $\ldots$''

Number footnotes separately in superscripts. Place the actual footnote at 
the bottom of the column in which it was cited. Do not put footnotes in the 
abstract or reference list. Use letters for table footnotes.

Unless there are six authors or more give all authors' names; do not use 
``et al.''. Papers that have not been published, even if they have been 
submitted for publication, should be cited as ``unpublished'' \cite{b4}. Papers 
that have been accepted for publication should be cited as ``in press'' \cite{b5}. 
Capitalize only the first word in a paper title, except for proper nouns and 
element symbols.

For papers published in translation journals, please give the English 
citation first, followed by the original foreign-language citation \cite{b6}.

\begin{thebibliography}{00}
\bibitem{b1} G. Eason, B. Noble, and I. N. Sneddon, ``On certain integrals of Lipschitz-Hankel type involving products of Bessel functions,'' Phil. Trans. Roy. Soc. London, vol. A247, pp. 529--551, April 1955.
\bibitem{b2} J. Clerk Maxwell, A Treatise on Electricity and Magnetism, 3rd ed., vol. 2. Oxford: Clarendon, 1892, pp.68--73.
\bibitem{b3} I. S. Jacobs and C. P. Bean, ``Fine particles, thin films and exchange anisotropy,'' in Magnetism, vol. III, G. T. Rado and H. Suhl, Eds. New York: Academic, 1963, pp. 271--350.
\bibitem{b4} K. Elissa, ``Title of paper if known,'' unpublished.
\bibitem{b5} R. Nicole, ``Title of paper with only first word capitalized,'' J. Name Stand. Abbrev., in press.
\bibitem{b6} Y. Yorozu, M. Hirano, K. Oka, and Y. Tagawa, ``Electron spectroscopy studies on magneto-optical media and plastic substrate interface,'' IEEE Transl. J. Magn. Japan, vol. 2, pp. 740--741, August 1987 [Digests 9th Annual Conf. Magnetics Japan, p. 301, 1982].
\bibitem{b7} M. Young, The Technical Writer's Handbook. Mill Valley, CA: University Science, 1989.
\end{thebibliography}

\end{document}
